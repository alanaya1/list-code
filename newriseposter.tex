
\documentclass[a0]{a0poster}
\usepackage[export]{adjustbox}

\input defs
\pagestyle{empty}
\setcounter{secnumdepth}{0}
\renewcommand{\familydefault}{\sfdefault}
\newcommand{\QED}{~~\rule[-1pt]{8pt}{8pt}}\def\qed{\QED}

\renewcommand{\reals}{{\mbox{\bf R}}}

\usepackage[absolute]{textpos}

\usepackage{fleqn,psfrag,wrapfig,tikz}

\usepackage[papersize={42in,36in}]{geometry}

\usepackage{graphics}
\usepackage{graphicx}


\usepackage{color}
\definecolor{Red}{rgb}{0.9,0.0,0.1}

\definecolor{bluegray}{rgb}{0.15,0.20,0.40}
\definecolor{bluegraylight}{rgb}{0.35,0.40,0.60}
\definecolor{gray}{rgb}{0.3,0.3,0.3}
\definecolor{lightgray}{rgb}{0.7,0.7,0.7}
\definecolor{darkblue}{rgb}{0.2,0.2,1.0}
\definecolor{darkgreen}{rgb}{0.0,0.5,0.3}

\renewcommand{\labelitemi}{\textcolor{bluegray}\textbullet}
\renewcommand{\labelitemii}{\textcolor{Red}{--}}

\setlength{\labelsep}{0.5em}



\let\Textsize\normalsize

\def\Head#1{\noindent{\LARGE\color{black} #1}\bigskip}
\def\LHead#1{\noindent{\LARGE\color{Red} #1}\bigskip}
\def\Subhead#1{\noindent{\large\color{Red} #1}\bigskip}
\def\Title#1{\noindent{\VeryHuge\color{black} #1}}

\TPGrid[40mm,40mm]{23}{18}      % 1 col of width 7, one of width 15,
                                % plus gap of width 1

\parindent=0pt
\parskip=0.2\baselineskip

\begin{document}

\begin{textblock}{2}(-0.6,0.2)
  \includegraphics[width=1.2\textwidth]{/users/Alex/Desktop/obsoletemat/engineeringlogo}
\end{textblock}


\begin{textblock}{23}(0,0)
\begin{center}

  \Title{Data Mining to Discover Hundreds of New Phase-Change Materials}
  \vspace{0.1in}

  {\Large Alex Anaya, Cameron Jones, Daniel Rehn$^*$, Evan
    Reed$^{\dag}$}

  $\;\;^*$ RISE mentor $\;\;^{\dag}$ Group PI
\end{center}
\author{Evan Reed}
\end{textblock}

% Uni logo in the top right corner. A&A in the bottom left. Gives a
% good visual balance, but you may want to change this depending upon
% the graphics that are in your poster.
%% \begin{textblock}{2}(0,10)
%% Your logo here
%% \includegraphics{/user...}
%% \end{textblock}

\begin{textblock}{2}(21.5,-0.15)
\includegraphics[width=1.1\textwidth]{/users/Alex/Desktop/obsoletemat/stanfordlogo}
\end{textblock}

\begin{center}
%\includegraphics[width=1.1\textwidth]{/users/Alex/Desktop/obsoletemat/engineeringlogo}
\end{center}

\begin{textblock}{7.0}(0,1.6)
\hrule\medskip
\Head{Introduction}
\vspace{-0.1in}
\begin{itemize}
\item Technologies of today and tomorrow depend on ways to store
  information reliably and quickly
\item Despite improvements in data storage technology, fundamentally new
  materials are needed

  \begin{figure}[!ht]\centering
    \includegraphics[width=0.9\textwidth]{/users/Alex/Desktop/obsoletemat/datastorageimage}
  \end{figure}

\item Phase change materials are pairs or groups of materials which have the same chemical formula,
  but different structures and properties. The most common example is diamond and coal, both are made completely of carbon
\item Reed group has undergone a large data mining operation to uncover as
  many prospective, two dimensional phase change materials as possible
\item Although Materials Project, a free to access website, contained documentation of
  these materials, filtering en masse had never been attempted
\item We created an algorithm to compile all of the materials in the Materials Project
  Database, filter through them, and generate a list
\item mono and few layer materials provide great promise for the future of data storing devices
  and will most likely outperform current devices.
\item Our list serves as an easy way to identify phase change materials, in an effort to
  expedite the process of researching these materials.
\end{itemize}

\medskip
\hrule\medskip
\Head{Background}
\vspace{-0.1in}
\begin{itemize}
\item Phase-change materials can be understood in terms of
  their crystal structure
\item Crystals can be defined by a \textbf{unit cell}: a
  three-dimensional box that, when stacked repeatedly in all
  directions, forms a solid
\item Structural and chemical properties of solids can be investigated
  in Density Functional Theory (DFT) codes by setting up a unit cell
  and using \emph{periodic boundary conditions}

  \begin{figure}[!ht]\centering
    \includegraphics[width=0.45\textwidth]{/users/Alex/Desktop/obsoletemat/structurediagram}
    \caption{\small Unit cell of Na$_4$Cl$_{4}$, also known as rock salt.}
  \end{figure}

\item Our algorithm searched almost 84000 DFT-computed materials to
 filter for phase-changing materials

\end{itemize}

\medskip
\hrule\medskip
\Head{POSCAR}
\vspace{-0.3in}
\begin{itemize}
\item VASP is a DFT code that uses a POSCAR file to define the unit cell
\end{itemize}

\begin{figure}[!ht]\centering
  \includegraphics[width=0.85\textwidth]{/users/Alex/Desktop/obsoletemat/poscar}
\end{figure}

\end{textblock}

%-------------------- Second Block --------------%
\begin{textblock}{15.0}(8,1.6)
\hrule\medskip
\Head{Algorithm}

The algorithm which we created is written in the python language and utilizes
various other extensions such as pymatgen, an open source python library for materials analysis,
to query the Materials project database for a desired set of characteristics, sort by various
criteria, such as band gap and chemical formula, then generate a list.

\begin{figure}[!ht]\centering
  \includegraphics[width=0.8\textwidth]{/users/Alex/Desktop/obsoletemat/codesnipet}
\end{figure}

\medskip
\hrule\medskip
\Head{Results}

\begin{figure}
  \includegraphics[width=0.4\textwidth]{/users/Alex/Desktop/obsoletemat/picset1}
  \includegraphics[width=0.5\textwidth]{/users/Alex/Desktop/obsoletemat/picset2}
  \label{fig:black}
\end{figure}


\begin{description}
\item[Left] A structural diagram of V3O7 one of the possible phase change materials
\item[Right] A structural diagram of Ca3Al2O6 one of the possible phase change materials


\end{description}

\medskip
\hrule\medskip
\Head{Conclusion}

We have developed an algorithm for parsing through the entirety of the Materials Project
database by selecting certain criteria to filter the database by, in an effort to identify
possible phase change for applications in data storage.

\medskip
\medskip
\hrule\medskip
\Head{Acknowledgements}

We thank the RISE program councelor, Maiken Bruhis, for her hard work
and dedication to the program. We also wish to thank other members of
the Reed group who have been supportive in our work. We also wish to
thank the Genentech Foundation for supporting our work here this summer.

%\hrule\medskip
\Head{References}

\small

[1] A. Jain, S.P. Ong, G. Hautier, W. Chen, W.D. Richards, S. Dacek, S. Cholia, D. Gunter, D. Skinner, G. Ceder, K.A. Persson, The Materials Project: A Materials Genome Approach to Accelerating Materials Innovation. APL Materials, doi:10.1063/1.4812323 (2013)

[2] Zimmermann, N. E., Horton, M. K., Jain, A., & Haranczyk, M. (2017). Assessing local structure motifs using order parameters for motif recognition, interstitial identification, and diffusion path characterization. Frontiers in Materials, 4, 34.

\includegraphics[width=5in,right]{/Users/Alex/Desktop/obsoletemat/genentech}

\end{textblock}






\end{document}
